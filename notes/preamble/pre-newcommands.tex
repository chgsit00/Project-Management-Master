% ------------------------------------------------------------------------
% LaTeX - Preambel ******************************************************
% ------------------------------------------------------------------------
% pre-newcommands
% ========================================================================
% ---- Hervorhebungen
% demo.tex Hervorhebungen
\newcommand{\env}[1]{\texttt{#1}}
\newcommand{\command}[1]{\texttt{#1}}
\newcommand{\package}[1]{\texttt{\itshape#1}}
\newcommand{\engl}[1]{(engl: \textit{#1})\xspace}

% todo
\newcommand{\todo}[1]{{\color{red}#1}\xspace}
\newcommand{\bv}{\todo{BV}} % Begriffsverzeichnis
\newcommand{\kap}{\todo{Kp}} % Kapitel

% TeX
\newcommand{\latex}{\LaTeX\xspace}
\newcommand{\tex}{\TeX\xspace}
\newcommand{\miktex}{MiK\TeX\xspace}
\newcommand{\bibtex}{Bib\TeX\xspace}

\newcommand{\led}{LEd\xspace}

\newcommand{\koma}{KOMA-Script\xspace}

% Internetseite
\newcommand{\www}[1]{\href{http://#1}{#1}}
\newcommand{\wwwhttp}[1]{\href{#1}{#1}}
\newcommand{\wwwlink}[1]{\footnote{\www{#1}}}

% Textauszeichnungen
\newcommand{\textemph}[1]{\textit{#1}} % Hervorheben
\newcommand{\textemphs}[1]{\textbf{#1}} % Hervorheben fett
\newcommand{\textqu}[1]{\enquote{#1}} % Anf�hrungszeichen
\newcommand{\tshortcut}[1]{\textit{#1}}
\newcommand{\textbutton}[1]{\textit{#1}}
\newcommand{\textmenu}[1]{\textit{#1}}
\newcommand{\textlst}[1]{\texttt{#1}} % Listings im Text
\newcommand{\requirements}[1]{\vspace{.1cm}\textbf{\sffamily#1}\\\noindent}
\newcommand{\usecase}[1]{\vspace{.1cm}\textbf{\sffamily Use Case \glqq #1\grqq}\\\noindent}
\newcommand{\langbeschreiubngusecase}[1]{\textbf{\sffamily Langbeschreibung f�r den Ablauf \glqq #1\grqq}\hfill\vspace{.4cm}}
\newcommand{\kommunikationsdiagramm}[1]{\textbf{\sffamily Kommunikationsdiagramm f�r den Ablauf \glqq #1\grqq}\hfill\vspace{.4cm}}
\newcommand{\kommunikationsdiagrammSchichten}[1]{\textbf{\sffamily Kommunikationsdiagramm mit Schichten f�r \glqq #1\grqq}\hfill\vspace{.4cm}}
%\newcommand{\textcode}[1]{\texttt{#1}\xspace} % 
%\newcommand{\texttask}[1]{\textit{#1}}
\newcommand\crule[3][black]{\textcolor{#1}{\rule{#2}{#3}}}

% ---- Abkuerzungen
\newcommand{\zB}{\mbox{z.\,B.}\xspace}
\newcommand{\ua}{\mbox{u.\,a.}\xspace}
\newcommand{\dah}{\mbox{d.\,h.}\xspace}
\newcommand{\uAe}{\mbox{u.\,�.}\xspace}

% ---- Listings
\newcommand{\lst}[1]{\lstinline$#1$} % geht nicht

\newcommand{\lstergibt}[1]{Ergibt:\newline{}}
%%%%%%%%%%%%%%%%%%%%%%%%%%%%%%%%%%%%%%%%%%%%%%%%%%%%%%%%%%%%%%%%%%%%%%%%%%%%%%
% ---- Querverweise
\newcommand{\refs}[1]{\mbox{(s.~\autoref{#1})}\xspace}
\newcommand{\refsauch}[1]{(s. auch \autoref{#1})\xspace}
\newcommand{\refn}[1]{\mbox{\autoref{#1}\xspace}} % normal

\newcommand{\refnp}[1]{\mbox{(\autopageref{#1})}\xspace}
\newcommand{\refp}[1]{Seite~\pageref{#1}\xspace}
%
\newcommand{\refk}[1]{Kapitel~\ref{#1}\xspace}
\newcommand{\refa}[1]{Abbildung~\ref{#1}\xspace}
\newcommand{\reft}[1]{Tabelle~\ref{#1}\xspace}
\newcommand{\reflst}[1]{Listing~\ref{#1}\xspace}
%%%%%%%%%%%%%%%%%%%%%%%%%%%%%%%%%%%%%%%%%%%%%%%%%%%%%%%%%%%%%%%%%%%%%%%%%%%%%%
% % ---- Literatur
% Verweise
\newcommand{\cites}[2]{(s. \cite[#1]{#2})\xspace}

% Bild aus Literaturv.
\newcommand{\cbild}[1]{(Bild~\cite{#1})\xspace}
%
%%%%%%%%%%%%%%%%%%%%%%%%%%%%%%%%%%%%%%%%%%%%%%%%%%%%%%%%%%%%%%%%%%%%%%%%%%%%%%
% % Escape innerhalb der Listings
\newcommand{\escapeBrown}[1]{\textcolor{brown}{#1}}
\newcommand{\escapeRed}[1]{\textcolor{red}{#1}}
\newcommand{\escapeCyan}[1]{\textcolor{cyan}{#1}}
%%%%%%%%%%%%%%%%%%%%%%%%%%%%%%%%%%%%%%%%%%%%%%%%%%%%%%%%%%%%%%%%%%%%%%%%%%%%%%
% ---- Namen der Links im Dokument
% ngerman (Babel-Paket) Namen umbenennen
\addto\captionsngerman{\renewcommand\figurename{Abb.}}
\addto\captionsngerman{\renewcommand\tablename{Tab.}}
\addto\captionsngerman{\renewcommand\lstlistingname{List.}}
%
%\addto\captionsngerman{\renewcommand\contentsname{Inhalt}}
%\addto\captionsngerman{\renewcommand\appendixname{Anhang}}
%\addto\captionsngerman{\renewcommand\lstlistlistingname{Listings}}
%
%\addto\extrasngerman{\def\partautorefname{Teil}}
\addto\extrasngerman{\def\chapterautorefname{Kap.}}
\addto\extrasngerman{\def\sectionautorefname{Kap.}}
\addto\extrasngerman{\def\subsectionautorefname{Kap.}}
\addto\extrasngerman{\def\subsubsectionautorefname{Kap.}}
\addto\extrasngerman{\def\subsectionautorefname{Kap.}}
\addto\extrasngerman{\def\paragraphautorefname{Kap.}}
\addto\extrasngerman{\def\subparagraphautorefname{Kap.}}
\addto\extrasngerman{\def\appendixautorefname{Kap.}}
%
\addto\extrasngerman{\def\figureautorefname{Abb.}}
\addto\extrasngerman{\def\tableautorefname{Tab.}}
\addto\extrasngerman{\def\equationautorefname{Gl.}}
\addto\extrasngerman{\def\theoremautorefname{Gl.}}
\addto\extrasngerman{\def\AMSnameautorefname{Gl.}}
\addto\extrasngerman{\def\pageautorefname{S.}}
%
%\addto\extrasngerman{\def\itemautorefname{Pkt.}}
%\addto\extrasngerman{\def\Hfootnoteautorefname{Fu�note}}
\addto\extrasngerman{\def\lstlistingautorefname{List.}}

