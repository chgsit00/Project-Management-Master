
\chapter{Introduction}
\label{sec:org4319e3a}

\section{Our company}
\label{sec:orgd179591}

\begin{center}
\includegraphics[width=.9\linewidth]{./content/dac-logo.png}
\end{center}
The DAC is a german organization dedicated to developing individual software
solutions for their customers that help them simplify their business processes
and improve productivity.

The founders of the company are Dennis, Antonio and Christoph, hence the name of
the company DAC, whereas this abbreviation also stands for "Digital Analog
Converter", which is a reference to the technical area in which we work.

We pay special attention to the usability of our products and have the desire to
provide our customers with a software solution customized to their specific
needs.

We believe that a software-assisted approach can help to make a company's
activities more environmentally friendly and efficient.

\section{Staff}
\label{sec:org5dfa11e}
\subsection{Dennis: Frontend Designer}
\label{sec:org6caf21f}
The role of the Frontend Designer is to create an eye-catching and easy-to-use
user interface for the job. Thereby it is important to be in close contact
with the customers and to ask for feedback based on concepts and prototypes.
\subsection{Tony: System Designer}
\label{sec:orgd672b29}
The role of the system designer is to ensure the reliable intercommunication between the backend and the hardware components.
Its job is to keep the costs for the hardware and implementation within the budget provided by the customer.
\subsection{Christoph: Software Architect}
\label{sec:orgd60e06e}
The role of the software architect is to divide the software solution into
smaller sub-components and to ensure the cooperation of these sub-components. It
is important to identify problems and evaluate special software solutions. His
task in our team is also to develop the backend system that the frontend is
accessing.

\section{iCare - our product}
\label{sec:org698e8a5}
\begin{center}
\includegraphics[width=.9\linewidth]{./content/iCare.jpeg}
\end{center}
Our product iCare, which we have developed as part of a customer order, is
supposed to be a software solution for a care home that is meant to make the
daily routine easier for all parties involved, as well as guaranteeing safety
and comfort.

It was taken into account that these care homes are mainly inhabited by elderly
people who are uneasy about new technologies. For this reason, we have set
ourselves the goal of ensuring that residents have as little involvement with
the system as possible and that most of the interaction is carried out by a
staff member.


\section{Aim}
\label{sec:orgaba15cc}
In addition to implementing customer requirements, we also focus on respecting
environmental standards. Our product is designed to provide patients with a
feel-good experience that excels in terms of high technological requirements and
efficient energy consumption.

\section{Objectives}
\label{sec:orgff22d59}


We want to ensure that 90 percent of our customers are extremely satisfied with
the software solution and that it is still in heavy use after 5 years. Also, 90
percent of the employees who use our software on a daily basis claim that their
daily work routine has been significantly simplified.

Customer satisfaction is important to us, but not everything can be taken into
account if the given budget of the contract is not sufficient. We aim to achieve
a 10 percent increase in sales each year.

We want to invest 30 percent of our revenue in training courses in order to
invest in the know-how of our employees.

Our product helps customers to better monitor their water and electricity
consumption and thus make an ecological contribution to environmental
protection. The aim is to reduce their consumption by 25 percent once the
product has been introduced.

