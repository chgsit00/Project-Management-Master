\chapter{Conclusion - Christoph Gschrey}
This project gave me the opportunity to gain some valuable experience in the field of project management. In particular, it showed how difficult it is to correctly estimate possible risks and problems in advance and to plan an appropriate reaction. An important lesson I have learned from the project is that one must always be aware of proportionality. To manage and document such a project with three people in such a short time, while all team members were working full time at the same time, was a challenge. That's why we had to make some cutbacks on certain points - especially in implementation. Unfortunately, we did not have the time to include all the requirements in our program in the end. While the time factor was a particular problem in the last month of the project, the group dynamics were consistently good. We helped each other when problems arose, but to a large extent worked individually on our tasks. Since we had precisely defined the intersections between our tasks beforehand, we had hardly any problems towards the end to combine the individual parts of the project and the individual parts of the documentation into a whole. All in all, I found the project neither particularly bad nor particularly good. Sometimes I had the feeling that certain things were completely pointless. For example, unlike the project plan, I didn't find the allocation of resources really helpful, since it was very difficult or even impossible to estimate something like this in our situation. If a task was to be completed, the person who is least busy at the moment simply takes over. I don't understand why we needed such a plan when the tasks could be managed much easier with Kanban at the same time.